\documentclass[onesided]{ccg-pset}

\course{Math 6230}
\psnum{12}
\author{Colton Grainger}
\date{\today}

\begin{document}
\maketitle

Problems from \emph{Introduction to Smooth Manifolds}, Chapter 11.
\setcounter{section}{11}

\begin{enumerate}

\setcounter{enumi}{29}
\item Say that $S$ is an immersed $k$-dimensional submanifold $\iota \colon S \to M$, and consider $f \in C^\infty(M)$. 
    Then the differential of $f$ restricted to $S$ is the pullback of the differential of $f$ under $\iota^*$, i.e., 
    \begin{equation}
        \label{1130}
        \dd{\paren{f\eval_S}} = \iota^*\paren{\dd{f}}.
    \end{equation}

    \begin{proof}
        Let $p \in S$. 
        Choose a slice chart $(U, (x^i))$ centered about $\iota(p)$ such that the first $k = \dim S$ coordinates $\iota^{-1} (x^i)$ parameterize $S \cap \iota^{-1}U$. 
        Identifying $S \cap \iota^{-1}U \inj \iota(S) \cap U$,
        both $f$ and $f \eval_S$ are determined (in the neighborhood $U$ of $p$) by coordinates by $(x^1, \ldots, x^k, x^{k+1}, \ldots, x^j)$.

        Now we evaluate the cotangent vectors on the LHS and RHS of \eqref{1130} at an arbitrary tangent vector $v_p \in T_p S \cong T_p (S\cap U)$, where $v_p = \sum\limits_{i=1}^{n} v^i \pdv{x_i}\eval_p$.
        On the LHS, $f\eval_S$ is constant with respect to $x^{k+1}, \ldots, x^n$, therefore 
        \begin{align*}
            \dd{\paren{f\eval_S}}(v_p) 
                &= \mqty[\pdv{f}{x^1}   & \cdots & \pdv{f}{x^k}]\eval_p    \mqty[\dd{x^1}\\ \vdots \\ \dd{x^k}] \eval_p 
                    \mqty[\pdv{x^1}   & \cdots & \pdv{x^n}]\eval_p    \mqty[v^1\\ \vdots \\ v^n] \\
                &= \mqty[\pdv{f}{x^1}   & \cdots & \pdv{f}{x^k}] \eval_p  [I_k \mid \vec 0] 
                    \mqty[v^1\\ \vdots \\ v^n] \\
                &= \mqty[\pdv{f}{x^1}   & \cdots & \pdv{f}{x^k}] \eval_p \mqty[v^1\\ \vdots \\ v^k].
        \end{align*}
        On the RHS, pulling back $v_p$ under $\iota^*$, we have
        \begin{equation*}
            \iota^*(\dd{f}_p)(v_p) = v_p f = \mqty[\pdv{f}{x^1} & \cdots & \pdv{f}{x^k}] \eval_p \mqty[v^1\\ \vdots \\ v^k].
        \end{equation*}
        This shows that the covectors on the LHS and RHS of \eqref{1130} agree at each point $p$ in $S$.
    \end{proof}

\setcounter{enumi}{5}
\item Let $M$ be a smooth $n$-dimensional manifold, let $W\subset M$ be an open subset of $M$, and consider $k$ smooth functions $y^{i} \colon W \to \R$ for $i = 1, \ldots, k$.

    \begin{enumerate}
        \item At a point $p \in W$, if $k = n$ and
            \begin{equation*}
                \dd{y^1}\eval_p, \ldots, \dd{y^k}\eval_p \qq{are linearly independent in $T_p^* M$,}
            \end{equation*}
            then
            \begin{equation*}
                \mqty[y^1\\\vdots \\ y^k] \qq{is a local coordinate system in a neighborhood $U \subset W$ of $p$.}
            \end{equation*}
            \begin{proof}
                Choose local coordinates $(x^j)$ about $p$. 
                Then each $y^i$ has differential w.r.t.~the $(x^j)$ given by
                \begin{equation*}
                    \mqty[\dd{y^1}\eval_p\\\vdots \\\dd{y^n}\eval_p]
                        = \mqty[\pdv{y^1}{x^1}\eval_p & \cdots & \pdv{y^1}{x^n}\eval_p \\
                                                      &        & \\
                        \pdv{y^n}{x^1}\eval_p         & \cdots & \pdv{y^n}{x^n}\eval_p]
                        \mqty[\dd{x^1}\eval_p \\ \vdots \\ \dd{x^n}\eval_p].
                \end{equation*}
                Our hypothesis is that the set $\set{\mqty[\pdv{y^i}{x^1}\eval_p & \cdots & \pdv{y^i}{x^n}\eval_p]}_{i=1}^n$ (vectors of evaluated partial derivatives) is linearly independent over $\R^n$.
                Therefore the linear transformation $\qty[\pdv{y^i}{x^j}\eval_p] = \bkt{\phi_*}$ is invertible, where $\phi$ is the map
                \begin{equation}
                    \phi \colon W \to \R^n \qq{such that}
                    q \overset{\phi}{\mapsto} \mqty[y^1(q)\\\vdots \\ y^n(q)].
                \end{equation}
                By the inverse function theorem, there's an open neighborhood 
                $U \subset W$ of $p$ such that $\phi^{-1} \circ \phi = \id \colon U \to U$, where $\phi^{-1}$ is smooth.
                So $\phi\eval_U$ is a diffeomorphism into $\R^n$.
                Therefore the $(y^i)$ are local coordinates on $U$.
            \end{proof}

        \item Similarly, if $p \in W$, $k < n$, and
            \begin{equation*}
                \dd{y^1}\eval_p, \ldots, \dd{y^k}\eval_p \qq{are linearly independent in $T_p^* M$,}
            \end{equation*}
            then
            \begin{equation*}
                \mqty[y^1\\\vdots \\ y^k] \qq{can be \emph{extended} to a local coordinates in a neighborhood $U \subset W$ of $p$.}
            \end{equation*}
            \begin{proof}
                Again, take local coordinates $(x^j)$ about $p$. 
                Each $y^i$ has differential w.r.t.~the $(x^j)$ given by
                \begin{equation}
                    \label{surjective}
                    \mqty[\dd{y^1}\eval_p\\\vdots \\\dd{y^k}\eval_p]
                        = \mqty[\pdv{y^1}{x^1}\eval_p & \cdots & \pdv{y^1}{x^n}\eval_p \\
                                                      &        & \\                   
                        \pdv{y^k}{x^1}\eval_p         & \cdots & \pdv{y^k}{x^n}\eval_p]
                        \mqty[\dd{x^1}\eval_p \\ \vdots \\ \dd{x^n}\eval_p].
                \end{equation}
                Assuming $k < n$ and $\set{\mqty[\pdv{y^i}{x^1}\eval_p & \cdots & \pdv{y^i}{x^n}\eval_p]}_{i=1}^k$ is linearly independent,
                there are precisely $n-k$ differentials $\dd{x^{j_\ell}}$ in the nullspace of $\qty[\pdv{y^i}{x^j}\eval_p]$ (\ref{surjective}).
                Define
                \begin{equation*}
                    \phi \colon W \to \R^n \qq{such that}
                    q \overset{\phi}{\mapsto} \mqty[y^1(q)\\\vdots \\ y^k(q)\\x^{j_1} \\ \vdots \\ x^{j_{n-k}}].
                \end{equation*}
                In particular, $k < n$ and \ref{surjective} imply that $\bkt{\phi_*}$ has full rank at $p$.
                Applying the inverse function theorem as before, there's a neighborhood $U$ such that $p \in U\subset W$ 
                for which $\phi\eval_U$ is a coordinate chart extending the $(y^j)$.
            \end{proof}
        \item Lastly, if $p \in W$, $k > n$, and $\dd{y^1}\eval_p, \ldots, \dd{y^k}\eval_p$ are linearly independent in $T_p^* M$ as before,
            then there are $n$ indices $i_\ell$ such that $\mqty[y^{i_1} & \vdots & y^{i_n}]$ is a chart on a neighborhood $U$ of $p$.
            \begin{proof}
                With local coordinates $(x^j)$ about $p$, each $y^i$ has differential w.r.t.~the $(x^j)$
                \begin{equation}
                    \label{injective}
                    \mqty[\dd{y^1}\eval_p\\\vdots \\\dd{y^k}\eval_p]
                        = \mqty[\pdv{y^1}{x^1}\eval_p & \cdots & \pdv{y^1}{x^n\eval_p} \\
                                                      &        & \\
                        \pdv{y^k}{x^1}\eval_p         & \cdots & \pdv{y^k}{x^n\eval_p}]
                        \mqty[\dd{x^1}\eval_p \\ \vdots \\ \dd{x^n}\eval_p],
                \end{equation}
                and our hypothesis is that the linear transformation $\qty[\pdv{y^i}{x^j}\eval_p]$ has rank $n$. 
                Choose $n$ differentials as a basis for the image of $\qty[\pdv{y^i}{x^j}\eval_p]$ (a subspace of $T^*_p M$), 
                call these $y^{i_\ell}$.
                Defined $\phi$ as the map
                \begin{align*}
                    \phi \colon W \to \R^n \qq{such that} q \overset{\phi}{\mapsto} \mqty[y^{i_1}(q)\\\vdots \\ y^{i_n}(q)].
                \end{align*}
                Then $k > n$ and equation \ref{injective} imply that $\bkt{\phi_*}$ has full rank at $p$.
                Apply the inverse function theorem to obtain a neighborhood $U$ such that $p \in U \subset W$ and $\phi\eval_U$ is a chart.
            \end{proof}
    \end{enumerate}

\setcounter{enumi}{10}
\item Say that $M$ is a smooth $n$-dim manifold, $C \subset M$ is embedded $n-k$ dim manifold, and $f \in C^\infty(M)$ is a smooth function.
    Suppose the restriction $f\eval_C$ has a (relative) maximum point at $p \in C$.
    Then fore any defining function $\Phi \colon U \to \R^k$ for $C$ on a neighborhood of $p$,
    there exist $k$ real numbers $\ell_1, \ldots, \ell_k$ such that 
    \begin{equation*}
        \dd{f}_p = \mqty[\ell_1 & \cdots & \ell_k] 
            \mqty[\dd{\Phi^1} \\ \vdots \\ \dd{\Phi^k}] \eval_p.
    \end{equation*}

    \begin{proof}
        If $C = M$, then $k = 0$, the defining function is trivial, and $\dd{f}_p = 0$. This follows because for any local chart $(x^i)$ centered at $p$, the path $f(\vec u t)$ has a local maximum point at $t =0$ for any unit vector $\vec u \in S^{n-1} \subset \R^n$, and therefore $\pdv{f}{x^i}  = 0$ for all $i = 1, \ldots, n$.

        %Say $C \neq M$, and let $\iota \colon C \inj M$ be the inclusion. 
        Let $\Phi \colon U \to \R^k$ be a local defining function for $C \cap U$.
        Because $\Phi$ is a submersion, the push-forwards $\Phi_*$ \emph{surjects} onto the tangent space $T_0 \R^k$. 
        Dually, the pullback $\Phi^*$ is a rank $k = \dim T_0\R^k$ \emph{injection} into the cotangent space $T_p M$.
        Because $\Phi \eval_C$ is constant, the image of the pullback $T^*_0\R^k \inj T^*_p M$ must be orthogonal to every cotangent vector in $T^*_p C$. So the cotangent space over $M$ splits
        \begin{equation}
            \label{split}
            T^*_p \R^k \oplus T^*_p C \xrightarrow{\cong} T^*_p M.
        \end{equation}
        
        Our hypothesis is that $p$ is a local maximum point of $f \eval_C$. 
        Consider that, for any smooth path $\gamma(t)$ in $C$ passing through $p$ at time $t =0$, 
        the composite $f(\gamma(t))$ has a local extremum at $t =0$. It follows that 
        \begin{equation*}
            \qq{every tangent vector $\gamma'(0)$ in $T_p C$ annihilates $f$} \implies \qq{the components of $\dd{f_p}$ in $T^*_p C$ are trivial.}
        \end{equation*}
        We have shown that the differential $\dd{f_p}$ is orthogonal to $T^*_p C$, so, by \ref{split}, $\dd{f_p}$ lies in the image of the pullback $\Phi^*$. The Lagrange multipliers $\ell_1, \ldots, \ell_k$ are just the coordinates of $\dd{f_p}$ in the covector space $T^* \R^k \inj T^*_p M$.
    \end{proof}

\end{enumerate}
\end{document}
