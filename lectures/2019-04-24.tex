\begin{quote}
\textit{%
    ``We're in the algebra of differential forms on a manifold, $\Omega(M)$.''
}
\end{quote}
% We talked a bit about the graduate linear algebra syllabus
% namely, in which course will tensor products be covered? 5000/6000 level? How many problems in the problem sets? Etc.
% It's not on the algebra syllabus.

\newcommand{\om}{\omega}

\begin{defn}[Multi-index notation]
    \label{defn:multi_index_notation}
    In local coordinates $(x^i)$, any $k$-form $\om$ can be expressed as
    \begin{equation*}
        \om = \sum\limits_{\paren{i_1, \ldots, i_k} \in I_n^k} 
            f_{\paren{i_1, \ldots, i_k}}(x) \dd{x}^{i_1} \wedge \cdots \wedge \dd{x}^{i_k}.
    \end{equation*}
    where $I_n^k$ is the product of tuples $\paren{1, \ldots, n} \times \paren{1, \ldots, k}$.
\end{defn}

We can compute the action of a covector $\om$ on $f_I(x)$ \TODO. (I got lost in the indices.)

\begin{equation*}
    f_I(x) = \om \mqty[\pdv{x^{i_1}} & \cdots & \pdv{x^{i_k}}] 
\end{equation*}

We'll start in $\R^n$. Let $\om$ be a $k$-form, of the form,
\begin{equation*}
    \om = \sum\limits_{\abs{I}= k} f_I(x) \dd{x}^{i_1} \wedge \cdots \wedge \dd{x}^{i_k}.
\end{equation*}
The exterior derivative of $\om$ is the $k+1$-form on $\R^n$ 
\begin{equation*}
   \dd{\om} \sum\limits_{\abs{I}= k} \dd{f_I} \wedge \dd{x}^{i_1} \wedge \cdots \wedge \dd{x}^{i_k}. 
\end{equation*}

In $\R^2$, we have $\dd{x \dd{y}} = \dd{x} \wedge \dd{y}$.

If $\om$ is a $1$-form, where $\om = f_i(x) \dd{x^i}$, then 
\begin{align*}
    \dd{\om} &= \dd{f_i} \wedge \dd{x^i}\\
        &= \paren{}
\end{align*}

\begin{ex}[Exterior derivative]
    \label{ex:exterior_derivative}
    
\end{ex}
