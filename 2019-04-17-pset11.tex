\documentclass[onesided]{ccg-pset}

\course{Math 6210}
\psnum{11}
\author{Colton Grainger}
\date{\today}

\begin{document}
\maketitle

Problems set from Chapter 9, Lee, \emph{Introduction to Smooth Manifolds.}
\begin{enumerate}

\item 9-1. 
    Suppose $M$ is a smooth manifold, $X$ is a vector field in $\X M$, and $\gamma$ is a maximum integral curve of $X$.
    \begin{enumerate}
        \item \textit{Either $\gamma$ is constant, $\gamma$ is injective, or $\gamma$ is periodic and nonconstant.}
        \item \textit{If $\gamma$ is periodic and nonconstant, then there's a unique positive number $T$ (called the \term{period} of $\gamma$) such that $\gamma(t) = \gamma(t')$ if and only if $t-t' \in T\Z$.}
        \item \textit{The image of $\gamma$ is an immersed submanifold of $M$, diffeomorphic to $\R$, $S^1$, or $\pt \approx \R^0$.}
    \end{enumerate}

    \newpage

    \newcommand{\dfgp}[1]{\mathrm{Diff}\paren{#1}} 
\item 9-7. 
    Say $M$ is a connected smooth manifold. \textit{The group of diffeomorphisms $\dfgp M$ acts transitively on $M$.}
    \begin{lem*}[]
        Let $B \subset \R^n$ be an open unit ball. If $p,q \in B$, there is a compactly supported smooth vector field $X \in \X B$ whose flow $\theta$ satisfies $\theta_1(p) = q$.
    \end{lem*}

    \newpage
\item 9-10.  
    \textit{For each of the following vector fields, find smooth coordinates in a neighborhood of $\mqty[1 & 0]^t$ such that the given vector field is a coordinate vector field.}
    % compute the flows! <ccg, 2019-04-16> %
    
    \begin{enumerate}
        \item  $V = y \pdv{x} + x \pdv{y}$.
        \item  $W = x \pdv{x} + 2y \pdv{y}$.
        \item  $X = x \pdv{x} - y \pdv{y}$.
        \item  $Y = x \pdv{y} + y \pdv{x}$.
    \end{enumerate}


    \newpage
\item 9-16. 
    \newcommand{\fake}[1]{\widetilde{#1}}
    \newcommand{\ldiff}[2]{\mathcal{L}_{#1}{#2}}
    \textit{Exhibit smooth vector fields $V$, $\fake V$, and $W$ on $\R^2$ such that $V = \fake V = \pdv*{x}$ along the $x$-axis but $\ldiff V W \neq \ldiff {\fake V} W$ at the origin.}

    \begin{note}[]
        This shows that it is necessary to know the vector field $V$ to compute $\ldiff V W$ at a point $p \in M$; it is not sufficient to just know the vector $V_p$, or even to know the values of $V$ along an integral curve of $V$.
    \end{note}



    \newpage
\item 9-18. 
    \newcommand{\flow}[1]{{\scriptstyle #1}\mathfrak{fl}}
    \newcommand{\dom}[1]{{\scriptstyle #1}B}
    Define vector fields $X, Y \in \X \R^2$ by 
    \[
        X = x \pdv{x} - y \pdv{y}, \quad Y = x \pdv{y} + y \pdv{x}.
    \]

    \begin{enumerate} 
        \item \textit{Compute the flows $\flow X$ and $\flow Y$ of $X$ and $Y$.}
        \item \textit{Find open intervals $\dom X$ and $\dom Y$ containing $0$ such that 
            \[
                \flow X _t \circ \flow Y _s  - \flow Y _s \circ \flow X _t \qq{is defined for all $\mqty[t & s]^t$ in $\dom X \times \dom Y$,}
            \]
        but nonzero at some time $\mqty[\bar{t} & \bar{s}]^t$.}
    \end{enumerate}

    \newpage
\item 9-19. 
    Consider $M = \R^3 \setminus \{z\text{-axis}\}$, the open submanifold of $\R^3$ formed by removing the $z$-axis. Define $V, W \in \X M$ by 
    \[
        V = \pdv{x} - \frac{y}{x^2 + y^2} \pdv{y}, \quad W= \pdv{y} + \frac{x}{x^2 + y^2} \pdv{z},
    \]
    and let $\flow V$, $\flow W$ be the flows of $V$ and $W$, respectively. \textit{Then $V$ and $W$ commute, but (caveat!) there exist a point $p \in M$ and times $s,t \in \R$ such that
    \[
        \paren{\flow V _t \circ \flow W _s}  (p) - \paren{\flow W _s \circ \flow V _t}  (p) \qq{is defined and nonzero.}
    \]}

\end{enumerate}
\end{document}
